%----------------------------------------------------------------------------------------
%	PACKAGES AND DOCUMENT CONFIGURATIONS
%----------------------------------------------------------------------------------------

\documentclass[12pt,a4paper]{article}
\usepackage[margin=1in]{geometry}
\usepackage{xeCJK}
\usepackage{listings}
\usepackage{graphicx} % Required for the inclusion of images
\usepackage{subfigure} % Required for the inclusion of images
\usepackage{natbib} % Required to change bibliography style to APA
\usepackage{amsmath} % Required for some math elements 
\usepackage{tcolorbox}
\usepackage{sectsty}
\usepackage{xcolor}
\usepackage{color}
\usepackage{fix-cm}
\usepackage{url}
\usepackage[nottoc,numbib]{tocbibind} % show refernce in toc
\usepackage[vlined,ruled,linesnumbered]{algorithm2e}
\usepackage{subfiles}
\usepackage{fancyhdr}
\pagestyle{fancy}
\usepackage{hyperref}
\usepackage{indentfirst}
\usepackage{underscore}
\hypersetup{
    colorlinks=true,
    linkcolor=cyan,
    filecolor=magenta,      
    urlcolor=cyan,
}  

%----------------------------------------------------------------------------------------
%	DOCUMENT INFORMATION
%----------------------------------------------------------------------------------------

\title{\textbf{计算机系统结构实验报告\ 实验1}} % Title
\author{赵一民\quad 518030910188\\ doctormin@sjtu.edu.cn } % Author name and email
\date{\today} % Date for the report


\definecolor{dkgreen}{rgb}{0,0.6,0}
\definecolor{mauve}{rgb}{0.58,0,0.50}

\lstset{
language=Verilog,                             % Code langugage
basicstyle=\ttfamily,             % Code font, Examples: \footnotesize, \ttfamily
keywordstyle=\color{purple},            % Keywords font ('*' = uppercase)
commentstyle=\em\color{gray},           % Comments font
numbers=left,                           % Line nums position
numberstyle=\tiny\ttfamily,             % Line-numbers fonts
stringstyle=\color{mauve},
stepnumber=1,                           % Step between two line-numbers
numbersep=5pt,                          % How far are line-numbers from code
frame=tb,                             % A frame around the code
aboveskip=3mm,
belowskip=3mm,
tabsize=4,                              % Default tab size
captionpos=b,                           % Caption-position = bottom
breaklines=true,                        % Automatic line breaking?
breakatwhitespace=false,                % Automatic breaks only at whitespace?
showspaces=false,                       % Dont make spaces visible
showstringspaces=false,
showtabs=false,                         % Dont make tabls visible
basewidth=0.5em,
morekeywords={},
columns=flexible
}

\def\inline{\lstinline[basicstyle=\ttfamily] 
}

\begin{document}
\maketitle % Insert the title, author and date

%----------------------------------------------------------------------------------------

\section{实验目的}
\begin{enumerate}
    \item 熟悉Xilinx逻辑设计工具Vivado的基本操作
    \item 掌握使用Verilog HDL进行简单的逻辑设计
    \item 使用功能仿真
\end{enumerate}
\section{原理分析}
\subsection{Flowing Light的设计}
\subsubsection{外部抽象}
\includegraphics[width=0.9\textwidth]{lab01-overview.jpg}
\begin{itemize}
    \item \texttt{clock}是时钟信号,在激励中可以设定周期
    \item \texttt{reset}是复位信号,可以将\texttt{counter}置为0以及将\texttt{led}置为\lstinline{8'h01}
    \item \texttt{counter}在代码中写成了\lstinline{reg [23:0] cnt_reg}, 每次上升沿会递增,因此范围是\lstinline{0 ~ 24'hffffff}
    \item \texttt{led}是唯一的输出信号,每当\texttt{counter}达到\lstinline{24'hffffff}时就加一,如果加到了\lstinline{8'h80}就在下一轮变回\lstinline{8'h01}
\pagebreak
\subsubsection{代码实现}
\begin{lstlisting}
module flowing_light(
        input clock,
        input reset,
        output [7:0] led
    );
    reg [23:0] cnt_reg;
    reg [7:0] light_reg;
    always @ (posedge clock) 
        begin
            if(reset)
                cnt_reg <= 0;
            else
                cnt_reg <= cnt_reg + 1;
        end
    always @ (posedge clock) 
        begin
            if(reset)
                light_reg <= 8'h01;
            else if (cnt_reg == 24'hffffff) 
                 begin
                     if(light_reg == 8'h80)
                        light_reg <= 8'h01;
                     else
                        light_reg <= light_reg << 1;
                 end
        end
    assign led = light_reg;
endmodule
\end{lstlisting}
\end{itemize}

\section{实验感想}

%----------------------------------------------------------------------------------------
\pagebreak

\end{document}
